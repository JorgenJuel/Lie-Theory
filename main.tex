\documentclass{article}
\usepackage[utf8]{inputenc}
\usepackage{amsthm}
\usepackage{amsmath}
\usepackage{amssymb}
 

\newtheorem{theorem}{Theorem}[section]
\theoremstyle{definition}
\newtheorem{definition}{Def}[section]
\newtheorem{proposition}{Prop}[section]
\newtheorem{corrollar}{Corr}[section]
\newtheorem{example}{Ex}[section]
 
\theoremstyle{remark}
\newtheorem*{remark}{Remark}

\title{Lie Theory}
\author{info }
\date{January 2019}

\begin{document}

\maketitle

\section{Introduction}
The general strategy is to "linearize" whatever we are working in. That is, reduce to linear algebra.

We will look at various forms of structures and their relationship with Lie Algebra. 

\begin{definition}
\textbf{An algebraic structure (theory)} is given by axioms for some sets $X_1, X_2, \ldots$ and sets of operations that are "n-ary". 
\begin{equation}
    X^n = \underbrace{X \times X \times \ldots X}_n \rightarrow X
\end{equation}
Theese axioms take the form that certains maps $X^m \rightarrow X$ are equal.
\end{definition}
\begin{definition}
The \underline{lower central series} of a group $G$ is defined inductively by $\Gamma _1 G = G$, and for $n \geq 2$, $\Gamma _n G = [G, \Gamma _{n-1} G]$ \\
This implies that $G = \Gamma _1 G \supseteq \Gamma _2 G \supseteq \ldots $
\end{definition}

\begin{proposition}
$\Gamma_n G $ is normal on $G$
\end{proposition}

\begin{proof}
$\Gamma _n G = [G, \Gamma _{n-1} G]$, which is generated by $[g,x]$ for $g\in G$, $x \in \Gamma _{n-1} G$. \\
$h \triangleright [g, x] = [ \underbrace{ h \triangleright g }_{\in G}, \underbrace{h \triangleright x}_{\in \Gamma _{n-1}G \text{ by induction}}]$
\end{proof}

\begin{definition}
$G/\Gamma _{n+1} G$ is \underline{the nilpotent quotient} of $G$ of class $n$. \\
if $n = 1$ then $G/[G, G] = G^{ab}$ (nilpotent of class 1 $\Leftrightarrow$ abelian) \\
\end{definition}

\begin{example}
if $G$ is free on $r$ generators, then: \\
$G/[G, G] = \mathbf{Z}^r $ (free abelian) \\
$G/\Gamma _{n+1} G = N_n^r $ (free nilpotent group of class $n$ on $r$ generators)
\end{example}

\begin{definition}
$G$ is nilpotent of class $n$ if $\Gamma _{n+1}G = 0$ \\
$\Rightarrow G$ is nilpotent of class $n$ for some $n$
\end{definition}

\begin{definition}
$G$ is residually nilpotent if $\bigcap \limits_n \Gamma _n G = \{ e \}$
\end{definition}

\begin{example}
$\Gamma _3 D_8 = \{e\} = \Gamma _3 Q_8$ (both nilpotent of class 2)
\end{example}

\begin{example}
$[S_3, [S_3, S_3] = [S_3, S_3] = (123)$ (Not nilpotent) 
\end{example}

\begin{theorem}[Hall-witt identity]
$[z \triangleright x, [y,z]] [x \triangleright y, [z, x]] [y \triangleright z, [x, y]] = e$ \\
$\implies [u, [v, w]] \subseteq [v, [w, u]] [w, [v, u]]$ for $u, v, w \trianglelefteq G$
\end{theorem}

\begin{proposition}
$[\Gamma _m G, \Gamma _n G] \subseteq \Gamma _{n+m} G$
\end{proposition}
\begin{proof}
Induction on m (or n) \\
\begin{flalign*}
    m = 1 \text{: } [G, \Gamma _n G] = \Gamma _{n+1} G \text{ by def.} \\
    m > 1 \text{: } [\Gamma _m G, \Gamma _n G] = [[G, \Gamma _{m-1} G], \Gamma _n G] \\
    \subseteq \underbrace{[G, [\Gamma _n G, \Gamma _{m-1} G]]}_{\subseteq \Gamma _ {n+m} G} \underbrace{[\Gamma _{m-1} G, [G, \Gamma_n]]}_{\subseteq \Gamma _ {n+m} G} = \Gamma _{n+m} G
\end{flalign*}
\end{proof}

\begin{proposition}
$\Gamma _n G /\Gamma _{n+1}$ G is abelian
\end{proposition}
\begin{proof}
$n = 1$: $G/[G, G]$ (abelian) \\
$[\Gamma _n G, \Gamma _n G] \subseteq \Gamma _{2n} G$ (by prop) $\subseteq \Gamma _{n+1} G $ (since $2n \geq n + 1$ for $n > 1$
\end{proof}

\begin{corrollar}
$y$ and $x \triangleright y$ represent the same element in $\Gamma _n G / \Gamma _{n+1} G$
\end{corrollar}

\begin{definition}
The associated graded of $G$, $gr(G)$: 
\begin{align*}
    gr(G) = \bigoplus\limits_{n=1}^\infty \frac{\Gamma _n G}{\Gamma _{n+1} G} = G^{\text{ab}} \oplus
\end{align*}
\end{definition}

\begin{example}
$gr(S_3) = \mathbf{Z}/2 \oplus 0 \oplus \ldots $ \\
$gr(D_8) = \mathbf{Z}/2 \times \mathbf{Z}/2 \oplus \mathbf{Z}/2 \oplus 0 \oplus \ldots$ \\
$gr(Q_8) = \mathbf{Z}/2 \times \mathbf{Z}/2 \oplus \mathbf{Z}/2 \oplus 0 \oplus \ldots$
\end{example}

However this is not so good for distinguishing different groups ($Q_8$ and $D_8$ have the same structure), therefore we define the "commutator operation" $[ , ]$ to propose a better invariant.

\begin{definition}
Commutator operator: 
\begin{align*}
    [,]: \frac{\Gamma _m G}{\Gamma _{m+1} G} \times \frac{\Gamma _n G}{\Gamma _{n+1} G} \longrightarrow \frac{\Gamma _{m + n} G}{\Gamma _{m+n + 1} G}
\end{align*}
Which is well defined and bilinear ($[x, yz] = [x, y][x, z]$ and $[xy, z] = [x, z] [y,z]$
\end{definition}

\begin{proposition}
The commutator operation satisfies:
\begin{enumerate}
    \item $[x, x] = 0$
    \item $[x, [y,z]] + [y, [z,x]] + [z, [x, y]] = 0$ (by hall-witt)
\end{enumerate}
\end{proposition}

\section{Groups}
\begin{definition}{Group} A set of elements $G$ together with three maps.
\begin{align*}
    G\times G &\longrightarrow G, & (g,h) &\longmapsto gh \\
    G &\longrightarrow G, & g &\longmapsto g^{-1} \\
    * &\longrightarrow G, & * &\longmapsto e
\end{align*}
Souch that 
\begin{align*}
    G\times G \times G &\longrightarrow G, & (g,h,k) &\longmapsto (gh)k = g(hk) \\
    G &\longrightarrow G, & g &\longmapsto gg^{-1} = e = g^{-1}g \\
    G & \longrightarrow G, & g &\longrightarrow ge = g  = eg
\end{align*}
\end{definition}
With the group $G$ we can define some maps:
\begin{description}
\item[Commutator] $[x,y] = xyx^{-1}y^{-1}$
\item[Conjucation] $y \triangleright y = xyx^{-x}$
\item $\langle x, y, z \rangle = xy^{-1}z$
\end{description}
\end{document}
